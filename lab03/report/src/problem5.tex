\CWHeader{Лабораторная работа \textnumero 3.5}
\CWProblem{
Вычислить определенный интеграл $F = \int_{X_0}^{X_1} y dx$, методами прямоугольников, трапеций, Симпсона с шагами 
$h_1, h_2$. Оценить погрешность вычислений, используя  Ме¬тод Рунге-Ромберга.

$$
y = \frac{x^2}{625 - x^4}  X_0 = 0, X_k = 4, h_1 = 1.0, h_2 = 0.5
$$
}

\section*{Описание}

\subsection*{Исходные данные}
Дана функция $y = f(x)$, заданная на отрезке $[X_0, X_1]$. Требуется вычислить определённый интеграл:
\[ F = \int_{X_0}^{X_1} f(x) dx \]

\subsection*{Методы решения}
Необходимо применить три классических метода численного интегрирования:

\begin{enumerate}
\item \textbf{Метод прямоугольников}:
\[ F_{\text{пр}} \approx h \sum_{i=0}^{n-1} f\left(x_i + \frac{h}{2}\right) \]
где $h = \frac{X_1 - X_0}{n}$

\item \textbf{Метод трапеций}:
\[ F_{\text{тр}} \approx \frac{h}{2} \left[f(X_0) + 2\sum_{i=1}^{n-1} f(x_i) + f(X_1)\right] \]

\item \textbf{Метод Симпсона} (парабол):
\[ F_{\text{симп}} \approx \frac{h}{3} \left[f(X_0) + 4\sum_{\text{неч}} f(x_i) + 2\sum_{\text{чет}} f(x_i) + f(X_1)\right] \]
\end{enumerate}

\subsection*{Параметры вычислений}
\begin{itemize}
\item Два различных шага интегрирования: $h_1$ и $h_2$ ($h_2 = \frac{h_1}{2}$)
\item Для каждого метода выполнить вычисления с обоими шагами
\end{itemize}

\subsection*{Оценка погрешности}
Применить \textbf{метод Рунге-Ромберга} для уточнения результата и оценки погрешности:
\[ F \approx F_h + \frac{F_h - F_{kh}}{k^p - 1} \]
где:
\begin{itemize}
\item $F_h$ -- значение интеграла с шагом $h$
\item $k$ -- коэффициент уменьшения шага ($k=2$)
\item $p$ -- порядок точности метода:
\begin{itemize}
\item $p=2$ для метода прямоугольников
\item $p=2$ для метода трапеций
\item $p=4$ для метода Симпсона
\end{itemize}
\end{itemize}

\subsection*{Требуемые результаты}
\begin{enumerate}
\item Значения интеграла для всех методов с шагами $h_1$ и $h_2$
\item Уточнённые значения по Рунге-Ромбергу
\item Оценки погрешности для каждого метода
\item Сравнительный анализ точности методов
\end{enumerate}

\subsection*{Особенности реализации}
\begin{itemize}
\item Для метода Симпсона количество отрезков должно быть чётным
\item Шаги $h_1$ и $h_2$ должны быть согласованы ($h_2 = h_1/2$)
\item При вычислениях учитывать все значащие цифры
\end{itemize}

\section*{Исходный код}

\begin{minted}{java}
package cat.mood;

import java.util.function.Function;

public class Integral {
    public static void main(String[] args) {
        Function<Double, Double> y = x -> Math.pow(x, 2) / (625 - Math.pow(x, 4));

        double x0 = 0.0;
        double x1 = 4.0;
        double h1 = 1.0;
        double h2 = 0.5;

        // Вычисление интеграла разными методами с шагом h1
        double rectH1 = rectangleMethod(y, x0, x1, h1);
        double trapH1 = trapezoidalMethod(y, x0, x1, h1);
        double simpH1 = simpsonMethod(y, x0, x1, h1);

        // Вычисление интеграла разными методами с шагом h2
        double rectH2 = rectangleMethod(y, x0, x1, h2);
        double trapH2 = trapezoidalMethod(y, x0, x1, h2);
        double simpH2 = simpsonMethod(y, x0, x1, h2);

        // Оценка погрешности и уточнение методом Рунге-Ромберга
        double rectRefined = rungeRombergRefined(rectH1, rectH2, h1, h2, 2);
        double trapRefined = rungeRombergRefined(trapH1, trapH2, h1, h2, 2);
        double simpRefined = rungeRombergRefined(simpH1, simpH2, h1, h2, 4);

        double rectError = Math.abs(rectRefined - rectH2);
        double trapError = Math.abs(trapRefined - trapH2);
        double simpError = Math.abs(simpRefined - simpH2);

        // Вывод результатов
        System.out.println("Метод прямоугольников:");
        System.out.printf("h=%.3f: F=%.8f\n", h1, rectH1);
        System.out.printf("h=%.3f: F=%.8f\n", h2, rectH2);
        System.out.printf("Уточнённое значение: %.8f\n", rectRefined);
        System.out.printf("Погрешность: %.8f\n\n", rectError);

        System.out.println("Метод трапеций:");
        System.out.printf("h=%.3f: F=%.8f\n", h1, trapH1);
        System.out.printf("h=%.3f: F=%.8f\n", h2, trapH2);
        System.out.printf("Уточнённое значение: %.8f\n", trapRefined);
        System.out.printf("Погрешность: %.8f\n\n", trapError);

        System.out.println("Метод Симпсона:");
        System.out.printf("h=%.3f: F=%.8f\n", h1, simpH1);
        System.out.printf("h=%.3f: F=%.8f\n", h2, simpH2);
        System.out.printf("Уточнённое значение: %.8f\n", simpRefined);
        System.out.printf("Погрешность: %.8f\n", simpError);
    }

    // Метод прямоугольников (средних)
    public static double rectangleMethod(Function<Double, Double> f, double a, double b, double h) {
        double sum = 0.0;
        double x = a + h / 2; // Средняя точка первого интервала
        while (x < b) {
            sum += f.apply(x);
            x += h;
        }
        return sum * h;
    }

    // Метод трапеций
    public static double trapezoidalMethod(Function<Double, Double> f, double a, double b, double h) {
        double sum = 0.5 * (f.apply(a) + f.apply(b));
        double x = a + h;
        while (x < b) {
            sum += f.apply(x);
            x += h;
        }
        return sum * h;
    }

    // Метод Симпсона
    public static double simpsonMethod(Function<Double, Double> f, double a, double b, double h) {
        double sum = f.apply(a) + f.apply(b);
        double x = a + h;
        boolean even = false;
        while (x < b) {
            sum += (even ? 2 : 4) * f.apply(x);
            x += h;
            even = !even;
        }
        return sum * h / 3;
    }

    // Уточнение значения интеграла методом Рунге-Ромберга-Ричардсона
    public static double rungeRombergRefined(double Ih, double Ih2, double h, double h2, int p) {
        return Ih2 + (Ih2 - Ih) / (Math.pow(h / h2, p) - 1);
    }
}
\end{minted}

\section*{Результат}

\begin{minted}{bash}
Метод прямоугольников:
h=1,000: F=0,04048897
h=0,500: F=0,04186829
Уточнённое значение: 0,04232806
Погрешность: 0,00045977

Метод трапеций:
h=1,000: F=0,04639504
h=0,500: F=0,04344201
Уточнённое значение: 0,04245766
Погрешность: 0,00098435

Метод Симпсона:
h=1,000: F=0,04302782
h=0,500: F=0,04245766
Уточнённое значение: 0,04241965
Погрешность: 0,00003801
\end{minted}

\section*{Вывод}

\subsection*{Сравнительные результаты методов}
\begin{table}[h]
\centering
\begin{tabular}{lccc}
\toprule
Метод & Значение при $h=1.0$ & Значение при $h=0.5$ & Уточнённое значение \\
\midrule
Прямоугольников & 0.04048897 & 0.04186829 & 0.04232806 \\
Трапеций & 0.04639504 & 0.04344201 & 0.04245766 \\
Симпсона & 0.04302782 & 0.04245766 & 0.04241965 \\
\bottomrule
\end{tabular}
\end{table}

\subsection*{Анализ точности методов}
\begin{enumerate}
\item \textbf{Сходимость результатов}:
\begin{itemize}
\item Все методы демонстрируют сходимость при уменьшении шага
\item Наибольшее изменение при уменьшении шага у метода прямоугольников ($\Delta = 0.001379$)
\item Наименьшее изменение у метода Симпсона ($\Delta = 0.000570$)
\end{itemize}

\item \textbf{Погрешности методов}:
\begin{itemize}
\item Наименьшая погрешность у метода Симпсона: $3.801 \times 10^{-5}$
\item Погрешность метода трапеций в 25 раз выше, чем у Симпсона
\item Метод прямоугольников показал промежуточную точность
\end{itemize}

\item \textbf{Эффективность уточнения}:
\begin{itemize}
\item Метод Рунге-Ромберга дал значительное уточнение для метода трапеций ($\Delta = 0.000984$)
\item Для метода Симпсона уточнение минимально, что свидетельствует о высокой исходной точности
\end{itemize}
\end{enumerate}

\pagebreak
