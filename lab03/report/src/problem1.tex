\CWHeader{Лабораторная работа \textnumero 3.1}

\CWProblem{
Используя таблицу значений $Y$ функции  $y=f(x)$, вычисленных в точках  $X_i, i = 0, ..., 3$ 
построить интерполяционные многочлены Лагранжа и Ньютона, проходящие через точки  ${X_i, 
 Y_i}$.  Вычислить значение погрешности интерполяции в точке  $X*$.

$$
y = \arcsin (x) + x
$$
а) $X_i = -0.4, -0.1, 0.2, 0.5$
б) $X_i = -0.4, 0, 0.2, 0.5$
$X* = 0.1$
}

\section*{Описание}

\subsection*{Постановка задачи}
Дана табличная функция в узлах $X_i$:
\[ (X_i, Y_i), \quad i=0,1,2,3 \]
где $Y_i = f(X_i)$. Требуется:
\begin{enumerate}
\item Построить интерполяционные многочлены:
\begin{itemize}
\item В форме Лагранжа $L_3(x)$
\item В форме Ньютона $P_3(x)$
\end{itemize}
\item Вычислить абсолютную погрешность в точке $X^*$:
\[ \Delta = |f(X^*) - P(X^*)|, \quad P \in \{L_3, P_3\} \]
\end{enumerate}

\subsection*{Интерполяционный многочлен Лагранжа}
Для $n+1$ узлов строится по формуле:
\[ L_n(x) = \sum_{i=0}^n Y_i \cdot \ell_i(x) \]
где базисные полиномы:
\[ \ell_i(x) = \prod_{\substack{j=0 \\ j\neq i}}^n \frac{x-X_j}{X_i-X_j} \]

Свойства:
\begin{itemize}
\item Точно проходит через все узлы: $L_n(X_i) = Y_i$
\item Степень многочлена равна $n$
\item Чувствителен к добавлению новых узлов
\end{itemize}

\subsection*{Интерполяционный многочлен Ньютона}
Строится через разделённые разности:
\[ P_n(x) = f[X_0] + \sum_{k=1}^n f[X_0,\ldots,X_k] \cdot \omega_k(x) \]
где:
\begin{itemize}
\item $\omega_k(x) = \prod_{i=0}^{k-1}(x-X_i)$
\item Разделённые разности:
\begin{align*}
f[X_i] &= Y_i \\
f[X_i,X_j] &= \frac{f[X_j]-f[X_i]}{X_j-X_i} \\
f[X_i,\ldots,X_k] &= \frac{f[X_{i+1},\ldots,X_k]-f[X_i,\ldots,X_{k-1}]}{X_k-X_i}
\end{align*}
\end{itemize}

Преимущества:
\begin{itemize}
\item Удобен при добавлении новых узлов
\item Позволяет оценить погрешность по первому отброшенному члену
\end{itemize}

\subsection*{Погрешность интерполяции}
Оценивается через остаточный член:
\[ R_n(x) = \frac{f^{(n+1)}(\xi)}{(n+1)!} \cdot \omega_{n+1}(x), \quad \xi \in [X_0,X_n] \]
где $\omega_{n+1}(x) = \prod_{i=0}^n (x-X_i)$.

На практике вычисляется как:
\[ \Delta(X^*) = |f(X^*) - P_n(X^*)|, \quad P_n \in \{L_n, P_n\} \]

\subsection*{Сравнение методов}
\begin{center}
\begin{tabular}{|l|c|c|}
\hline
Характеристика & Лагранж & Ньютон \\
\hline
Чувствительность к новым узлам & Требует пересчёта & Частичный пересчёт \\
Вычислительная сложность & $O(n^2)$ & $O(n^2)$ \\
\hline
\end{tabular}
\end{center}

\section*{Исходный код}

\begin{minted}{java}
package cat.mood;

import java.util.function.Function;

public class A {
    public record Pair(Function<Double, Double> first, Function<String, String> second) {}

    static double omega(int n, int idx, double[] x) {
        double result = 1;
        for (int i = 0; i < n; ++i) {
            if (i != idx) {
                result *= x[idx] - x[i];
            }
        }

        return result;
    }

    public static double[][] difference(double[] x, double[] y) {
        int n = x.length;
        double[][] table = new double[n][n];


        for (int i = 0; i < n; i++) {
            table[i][0] = y[i];
        }

        for (int j = 1; j < n; j++) {
            for (int i = 0; i < n - j; i++) {
                table[i][j] = (table[i + 1][j - 1] - table[i][j - 1]) / (x[i + j] - x[i]);
            }
        }

        return table;
    }

    public static Pair lagrange(double[] x, double[] y) {
        int n = x.length;
        double[] w = new double[n];
        for (int i = 0; i < n; ++i) {
            w[i] = omega(n, i, x);
        }

        Function<String, String> fs = str -> {
            StringBuilder sb = new StringBuilder();
            for (int i = 0; i < n; ++i) {
                if (y[i] / w[i] > 0) {
                    sb.append("+");
                }
                sb.append(y[i] / w[i]);
                for (int j = 0; j < n; ++j) {
                    if (i != j) {
                        sb.append("(x ");
                        if (x[j] < 0) {
                            sb.append("+ ").append(x[j]);
                        } else {
                            sb.append("- ").append(x[j]);
                        }
                        sb.append(")");
                    }
                }
                sb.append(" ");
            }
            return sb.toString();
        };

        Function<Double, Double> fd = t -> {
            double f = 0;
            for (int i = 0; i < n; ++i) {
                double fi = y[i] / w[i];
                for (int j = 0; j < n; ++j) {
                    if (i != j) {
                        fi *= t - x[j];
                    }
                }
                f += fi;
            }

            return f;
        };
        return new Pair(fd, fs);
    }

    public static Pair newton(double[] x, double[] y) {
        double[][] d = difference(x, y);
        int n = x.length;

        Function<String, String> fs = str -> {
            StringBuilder sb = new StringBuilder();
            sb.append(y[0]);
            for (int i = 1; i < n; ++i) {
                if (d[0][i] > 0) {
                    sb.append(" + ");
                } else {
                    sb.append(" - ");
                }
                sb.append(Math.abs(d[0][i]));
                for (int j = 0; j < i; ++j) {
                    sb.append("(x ");
                    if (x[i] < 0) {
                        sb.append("+ ");
                    } else {
                        sb.append("- ");
                    }
                    sb.append(Math.abs(x[j])).append(")");
                }
            }
            return sb.toString();
        };

        Function<Double, Double> fd = t -> {
            double f = y[0];
            for (int i = 1; i < n; ++i) {
                double fi = d[0][i];
                for (int j = 0; j < i; ++j) {
                    fi *= t - x[j];
                }
                f += fi;
            }

            return f;
        };

        return new Pair(fd, fs);
    }

    static void solve(double[] x, double[] y, double t, Function<Double, Double> f) {
        var L = lagrange(x, y);
        System.out.println("Многочлен Лагранжа:");
        System.out.println(L.second.apply("x"));
        System.out.println("Многочлен Лангранжа в точке x = " + t + ": " + L.first.apply(t));
        System.out.println("Функция в точке x = " + t + ": " + f.apply(t));
        System.out.println("Погрешность: " + Math.abs(f.apply(t) - L.first.apply(t)));

        System.out.println();

        var N = newton(x, y);
        System.out.println("Многочлен Ньютона:");
        System.out.println(N.second.apply("x"));
        System.out.println("Многочлен Ньютона в точке x = " + t + ": " + N.first.apply(t));
        System.out.println("Функция в точке x = " + t + ": " + f.apply(t));
        System.out.println("Погрешность: " + Math.abs(f.apply(t) - N.first.apply(t)));
    }

    public static void main(String[] args) {
        Function<Double, Double> f = x -> Math.asin(x) + x;
        double[] x1 = {-0.4, -0.1, 0.2, 0.5};
        double[] x2 = {-0.4, 0, 0.2, 0.5};
        double[] y1 = new double[x1.length];
        double[] y2 = new double[x2.length];
        for (int i = 0; i < x1.length; ++i) {
            y1[i] = f.apply(x1[i]);
            y2[i] = f.apply(x2[i]);
        }

        double t = 0.1;

        System.out.println("а)");
        solve(x1, y1, t, f);
        System.out.println("б)");
        solve(x2, y2, t, f);
    }
}
\end{minted}

\section*{Результат}

\begin{minted}{bash}
а)
Многочлен Лагранжа:
+5.009363247330172(x + -0.1)(x - 0.2)(x - 0.5) -3.70680409558444(x + -0.4)(x - 0.2)(x - 0.5) -7.4325540887098285(x + -0.4)(x + -0.1)(x - 0.5) +6.3185109604833265(x + -0.4)(x + -0.1)(x - 0.2) 
Многочлен Лангранжа в точке x = 0.1: 0.2000558780105125
Функция в точке x = 0.1: 0.2001674211615598
Погрешность: 1.1154315104730528E-4

Многочлен Ньютона:
-0.8115168460674881 + 2.037831416353094(x + 0.4) - 0.05457823863354249(x - 0.4)(x - 0.1) + 0.18851602351922966(x - 0.4)(x - 0.1)(x - 0.2)
Многочлен Ньютона в точке x = 0.1: 0.20005587801051233
Функция в точке x = 0.1: 0.2001674211615598
Погрешность: 1.1154315104747181E-4
б)
Многочлен Лагранжа:
+3.757022435497629(x - 0.0)(x - 0.2)(x - 0.5) 0.0(x + -0.4)(x - 0.2)(x - 0.5) -11.148831133064744(x + -0.4)(x - 0.0)(x - 0.5) +7.582213152579992(x + -0.4)(x - 0.0)(x - 0.2) 
Многочлен Лангранжа в точке x = 0.1: 0.20009364664038548
Функция в точке x = 0.1: 0.2001674211615598
Погрешность: 7.377452117432459E-5

Многочлен Ньютона:
-0.8115168460674881 + 2.02879211516872(x - 0.4) - 0.03667085202844348(x - 0.4)(x - 0.0) + 0.19040445501287723(x - 0.4)(x - 0.0)(x - 0.2)
Многочлен Ньютона в точке x = 0.1: 0.20009364664038537
Функция в точке x = 0.1: 0.2001674211615598
Погрешность: 7.377452117443561E-5
\end{minted}

\section*{Вывод}

На основании проведённых вычислений можно сделать следующие выводы:

\textbf{Точность интерполяции}:
\begin{itemize}
\item В обоих случаях (a и b) методы Лагранжа и Ньютона дали практически идентичные результаты
\item Погрешность интерполяции в точке $x=0.1$ составила:
\begin{itemize}
\item Для случая (a): $\approx 1.115 \times 10^{-4}$
\item Для случая (b): $\approx 7.377 \times 10^{-5}$
\end{itemize}
\end{itemize}

\textbf{Сравнение методов}:
\begin{itemize}
\item Оба метода показали сопоставимую точность в заданной точке
\item Значения многочленов Лагранжа и Ньютона в точке $x=0.1$ совпадают с точностью до $10^{-16}$
\item Разница в погрешностях между методами незначительна ($\sim 10^{-19}$)
\end{itemize}

\textbf{Анализ результатов}:
\begin{itemize}
\item Случай (b) демонстрирует меньшую погрешность, что может быть связано с более удачным расположением узлов интерполяции
\item Полученная погрешность порядка $10^{-4}-10^{-5}$ свидетельствует о хорошей точности обоих методов
\item Небольшие различия в результатах могут быть обусловлены особенностями округления при вычислениях
\end{itemize}

\textbf{Практические рекомендации}:
\begin{itemize}
\item Для данной задачи оба метода интерполяции показали себя как эффективные инструменты
\item Выбор между методами может основываться на:
\begin{itemize}
\item Удобстве реализации (метод Ньютона проще модифицировать при добавлении новых узлов)
\item Вычислительной эффективности
\end{itemize}
\item Для достижения максимальной точности рекомендуется:
\begin{itemize}
\item Оптимизировать расположение узлов интерполяции
\item Учитывать поведение интерполируемой функции
\end{itemize}
\end{itemize}

Таким образом, проведённые вычисления подтвердили теоретические положения о равнозначной точности интерполяционных многочленов 
Лагранжа и Ньютона при одинаковых условиях и продемонстрировали их практическую применимость для решения задач аппроксимации.

\pagebreak
