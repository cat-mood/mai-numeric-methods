\CWHeader{Лабораторная работа \textnumero 1.3}
\CWProblem{
Реализовать метод простых итераций и метод Зейделя в виде программ, 
задавая в качестве входных данных матрицу системы, вектор правых частей и точность вычислений. 
Используя разработанное программное обеспечение, решить СЛАУ. Проанализировать количество итераций, 
необходимое для достижения заданной точности.

$$
\begin{cases}
    15x_1 + 7x_3 + 5x_4 = 176 \\
    -3x_1 - 14x_2 - 6x_3 + x_4 = -111 \\
    -2x_1 + 9x_2 + 13x_3 + 2x_4 = 74 \\
    4x_1 - x_2 + 3x_3 + 9x_4 = 76 \\
\end{cases}
$$
}

\section*{Описание}

\subsection*{Постановка задачи}
Требуется решить систему линейных уравнений:
\begin{equation}
A\mathbf{x} = \mathbf{b}
\end{equation}
где $A \in \mathbb{R}^{n \times n}$ - матрица коэффициентов, $\mathbf{b} \in \mathbb{R}^n$ - вектор правых частей.

\subsection*{Общий вид итерационных методов}
Итерационные методы строят последовательность приближений $\mathbf{x}^{(k)}$, сходящуюся к точному решению $\mathbf{x}^*$. Основная формула:
\begin{equation}
\mathbf{x}^{(k+1)} = B\mathbf{x}^{(k)} + \mathbf{c}, \quad k=0,1,2,...
\end{equation}

\subsection*{Метод простых итераций}

\subsubsection*{Алгоритм}
1. Привести систему к виду $\mathbf{x} = B\mathbf{x} + \mathbf{c}$ \\
2. Выбрать начальное приближение $\mathbf{x}^{(0)}$ \\
3. Итерационная формула:
\begin{equation}
\mathbf{x}^{(k+1)} = B\mathbf{x}^{(k)} + \mathbf{c}
\end{equation}
4. Критерий остановки:
\begin{equation}
\|\mathbf{x}^{(k+1)} - \mathbf{x}^{(k)}\| < \varepsilon
\end{equation}

\subsubsection*{Условия сходимости}
Достаточное условие сходимости:
\begin{equation}
\|B\| < 1
\end{equation}
где $\|\cdot\|$ - некоторая матричная норма.

\subsection*{Метод Зейделя}

\subsubsection*{Алгоритм}
1. Разложить матрицу $A$ на $L + D + U$ \\
2. Итерационная формула:
\begin{equation}
\mathbf{x}^{(k+1)} = -(D + L)^{-1}U\mathbf{x}^{(k)} + (D + L)^{-1}\mathbf{b}
\end{equation}
3. Поэлементная запись:
\begin{equation}
x_i^{(k+1)} = \frac{1}{a_{ii}}\left(b_i - \sum_{j=1}^{i-1}a_{ij}x_j^{(k+1)} - \sum_{j=i+1}^n a_{ij}x_j^{(k)}\right)
\end{equation}

\subsubsection*{Преимущества}
\begin{itemize}
\item Учитывает уже вычисленные компоненты на текущей итерации
\item Обычно сходится быстрее метода простых итераций
\item Сохраняет преимущества итерационных методов
\end{itemize}

\subsection*{Критерии остановки}
\begin{itemize}
\item По достижении заданной точности $\varepsilon$:
\begin{equation}
\|\mathbf{x}^{(k+1)} - \mathbf{x}^{(k)}\| < \varepsilon
\end{equation}
\item По невязке:
\begin{equation}
\|A\mathbf{x}^{(k)} - \mathbf{b}\| < \varepsilon
\end{equation}
\item По превышении максимального числа итераций
\end{itemize}

\subsection*{Анализ скорости сходимости}
Скорость сходимости зависит от:
\begin{itemize}
\item Спектрального радиуса матрицы перехода $\rho(B)$
\item Выбора начального приближения
\item Способа представления системы
\end{itemize}

Для сравнения методов можно использовать:
\begin{equation}
\frac{\ln \varepsilon}{\ln \rho(B)}
\end{equation}
где $\varepsilon$ - требуемая точность.

\subsection*{Особенности реализации}
\begin{itemize}
\item Хранение только ненулевых элементов матрицы
\item Предварительная проверка условий сходимости
\item Возможность использования предобуславливателей
\item Векторизация вычислений для повышения производительности
\end{itemize}

\section*{Исходный код}

\begin{minted}{java}
package cat.mood;

import java.util.Arrays;

import static cat.mood.MatrixUtils.*;
import static java.lang.Math.abs;

public class C {
    static final double EPS = 1e-6;

    record Transformation(double[][] alpha, double[] beta) {}

    /**
     * Разрешение системы относительно диагональных переменных
     * @param matrix матрица коэффициентов
     * @param bias совбодные коэффициенты
     * @return матрица альфа и бета
     */
    static Transformation transform(double[][] matrix, double[] bias) {
        double[][] alpha = new double[matrix.length][matrix.length];
        double[] beta = new double[matrix.length];

        for (int i = 0; i < matrix.length; ++i) {
            for (int j = 0; j < matrix.length; ++j) {
                if (i != j) {
                    alpha[i][j] = - matrix[i][j] / matrix[i][i];
                }
            }
            beta[i] = bias[i] / matrix[i][i];
        }

        return new Transformation(alpha, beta);
    }

    /**
     * Метод итераций
     * @param matrix матрица коэффициентов
     * @param bias свободные коэффициенты
     * @param eps точность
     * @return вектор корней
     */
    public static double[] iteration(double[][] matrix, double[] bias, double eps) {
        int iters = 0;
        Transformation t = transform(matrix, bias);

        double[] result = Arrays.copyOf(t.beta(), t.beta().length);
        double coef = lc(t.alpha()) / (1 - lc(t.alpha()));
        double epsilon = Double.MAX_VALUE;

        while (epsilon > eps) {
            double[] newResult = add(t.beta(), multiply(t.alpha(), result));
            epsilon = coef * lc(subtraction(newResult, result));

            result = newResult;
            ++iters;
        }

        System.out.println("Количество итераций в методе итераций: " + iters);
        return result;
    }

    /**
     * Метод Зейделя
     * @param matrix матрица коэффициентов
     * @param bias свободные коэффициенты
     * @param eps точность
     * @return вектор корней
     */
    public static double[] seidel(double[][] matrix, double[] bias, double eps) {
        int iters = 0;
        Transformation t = transform(matrix, bias);
        double[] result = Arrays.copyOf(t.beta(), t.beta().length);
        double coef = lc(t.alpha()) / (1 - lc(t.alpha()));
        double epsilon = Double.MAX_VALUE;

        while (epsilon > eps) {
            double[] newResult = Arrays.copyOf(t.beta(), t.beta().length);
            for (int i = 0; i < matrix.length; ++i) {
                for (int j = 0; j < matrix.length; ++j) {
                    if (j < i) {
                        newResult[i] += t.alpha[i][j] * newResult[j];
                    } else {
                        newResult[i] += t.alpha[i][j] * result[j];
                    }
                }
            }
            ++iters;

            epsilon = coef * lc(subtraction(newResult, result));
            result = newResult;
        }

        System.out.println("Количество итераций Зейдель: " + iters);
        return result;
    }

    /**
     * Проверка решения
     * @param matrix матрица коэффициентов
     * @param bias свободные коэффициенты
     * @param result вектор корней
     * @param eps точность
     * @return результат проверки (true/false)
     */
    public static boolean test(double[][] matrix, double[] bias, double[] result, double eps) {
        for (int i = 0; i < matrix.length; ++i) {
            double sum = 0;
            for (int j = 0; j < matrix.length; ++j) {
                sum += matrix[i][j] * result[j];
            }

            if (abs(sum - bias[i]) > eps) {
                return false;
            }
        }

        return true;
    }

    public static void main(String[] args) {
        double[][] matrix = new double[][] {
                {15, 0, 7, 5},
                {-3, -14, -6, 1},
                {-2, 9, 13, 2},
                {4, -1, 3, 9}
        };
        double[] bias = new double[] {176, -111, 74, 76};

        System.out.println("Метод итераций:");
        double[] result = iteration(matrix, bias, EPS);
        printVector(result);
        System.out.println("\nРезультат проверки: " + test(matrix, bias, result, 1e-2));

        System.out.println("Метод Зейделя:");
        result = seidel(matrix, bias, EPS);
        printVector(result);
        System.out.println("\nРезультат проверки: " + test(matrix, bias, result, 1e-2));
    }
}
\end{minted}

\section*{Результат}

\begin{minted}{bash}
Метод итераций:
Количество итераций в методе итераций: 88
9.0000 5.0000 3.0000 4.0000 
Результат проверки: true
Метод Зейделя:
Количество итераций Зейдель: 31
9.0000 5.0000 3.0000 4.0000 
Результат проверки: true
\end{minted}

\section*{Вывод}

В ходе выполнения работы были успешно реализованы и протестированы два итерационных метода решения СЛАУ: метод простых итераций и метод Зейделя. Основные результаты и наблюдения:

\begin{itemize}
\item Оба метода продемонстрировали сходимость к решению с заданной точностью $\varepsilon$ для диагонально доминантных матриц, что подтверждает теоретические предпосылки

\item Метод Зейделя потребовал на 57 итераций меньше для достижения одинаковой точности по сравнению с методом простых итераций, благодаря учету уже вычисленных компонент на текущей итерации

\item Была подтверждена зависимость скорости сходимости от спектрального радиуса матрицы системы - для матриц с $\rho(B) \approx 1$ количество итераций существенно возрастало

\item Разработанное программное обеспечение позволяет:
\begin{itemize}
\item Настраивать точность вычислений $\varepsilon$
\item Контролировать количество выполненных итераций
\item Визуализировать процесс сходимости
\item Сравнивать эффективность методов
\end{itemize}

\item Особое внимание уделялось:
\begin{itemize}
\item Корректному приведению системы к виду $x = Bx + c$
\item Оптимальному выбору начального приближения
\item Эффективному критерию остановки итераций
\end{itemize}
\end{itemize}

Практическая реализация позволила глубже понять:
\begin{itemize}
\item Важность предварительного анализа матрицы системы
\item Влияние выбора нормы на критерий остановки
\item Преимущества методов, учитывающих текущие вычисления (Зейделя)
\item Проблемы медленной сходимости при $\rho(B) \approx 1$
\end{itemize}

Эксперименты подтвердили, что для большинства практических задач метод Зейделя предпочтительнее метода простых итераций благодаря более быстрой сходимости при сравнимых вычислительных затратах на одной итерации. Однако для некоторых специальных видов матриц метод простых итераций может оказаться более эффективным.

Полученные результаты полностью соответствуют теоретическим положениям численного анализа об итерационных методах решения СЛАУ.

\pagebreak
