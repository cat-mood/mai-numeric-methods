\CWHeader{Лабораторная работа \textnumero 1.2}
\CWProblem{
Реализовать метод прогонки в виде программы, задавая в качестве входных данных ненулевые элементы 
матрицы системы и вектор правых частей. Используя разработанное программное обеспечение, решить СЛАУ с 
трехдиагональной матрицей.

$$
\begin{cases}
    10x_1 - x_2 = 16 \\
    -8x_1 + 16x_2 + x_3 = -110 \\
    6x_2 - 16x_3 + 6x_4 = 24 \\
    -8x_3 + 16x_4 - 5x_5 = -3 \\
    5x_4 - 13x_5 = 87 \\
\end{cases}
$$
}

\section*{Описание}

\subsection*{Постановка задачи}
Рассмотрим систему линейных алгебраических уравнений (СЛАУ) вида:
\begin{equation}
A\mathbf{x} = \mathbf{f}
\end{equation}
где $A$ - трехдиагональная матрица размера $n \times n$:
\[
A = \begin{pmatrix}
b_1 & c_1 & 0   & \cdots & 0 \\
a_2 & b_2 & c_2 & \ddots & \vdots \\
0   & a_3 & b_3 & \ddots & 0 \\
\vdots & \ddots & \ddots & \ddots & c_{n-1} \\
0 & \cdots & 0 & a_n & b_n
\end{pmatrix}, \quad
\mathbf{x} = \begin{pmatrix} x_1 \\ x_2 \\ \vdots \\ x_n \end{pmatrix}, \quad
\mathbf{f} = \begin{pmatrix} f_1 \\ f_2 \\ \vdots \\ f_n \end{pmatrix}
\]

\subsection*{Алгоритм метода прогонки}
Метод прогонки состоит из двух этапов - прямого и обратного хода.

\subsubsection*{Прямой ход (вычисление прогоночных коэффициентов)}
1. Вычисляем начальные коэффициенты:
\begin{align*}
\alpha_1 &= -\frac{c_1}{b_1} \\
\beta_1 &= \frac{f_1}{b_1}
\end{align*}

2. Для $i = 2, \ldots, n-1$ вычисляем:
\begin{align*}
\alpha_i &= -\frac{c_i}{b_i + a_i\alpha_{i-1}} \\
\beta_i &= \frac{f_i - a_i\beta_{i-1}}{b_i + a_i\alpha_{i-1}}
\end{align*}

3. Для $i = n$:
\[
\beta_n = \frac{f_n - a_n\beta_{n-1}}{b_n + a_n\alpha_{n-1}}
\]

\subsubsection*{Обратный ход (нахождение решения)}
1. Последняя компонента решения:
\[
x_n = \beta_n
\]

2. Для $i = n-1, \ldots, 1$ вычисляем:
\[
x_i = \alpha_i x_{i+1} + \beta_i
\]

\subsection*{Условия применимости}
Метод прогонки корректен и устойчив, если:
\begin{itemize}
\item Матрица $A$ имеет диагональное преобладание:
\[
|b_i| \geq |a_i| + |c_i|, \quad i = 1,\ldots,n
\]
(причем хотя бы для одного $i$ неравенство строгое)
\item Либо матрица $A$ симметрична и положительно определена
\end{itemize}

\subsection*{Вычислительная сложность}
Метод прогонки требует:
\begin{itemize}
\item $8n - 7$ арифметических операций
\item $3n - 2$ ячеек памяти (для хранения коэффициентов)
\end{itemize}
что существенно эффективнее методов для плотных матриц ($O(n^3)$ операций).

\subsection*{Особенности реализации}
При программной реализации следует:
\begin{itemize}
\item Проверять условие диагонального преобладания
\item Контролировать знаменатели в формулах для $\alpha_i$
\item Обеспечивать хранение только ненулевых элементов
\end{itemize}

\section*{Исходный код}

\begin{minted}{java}
package cat.mood;

import java.util.ArrayList;
import java.util.List;

import static cat.mood.MatrixUtils.printVector;
import static java.lang.Math.abs;

public class B {
    static final double EPS = 1e-6;

    /**
     * Решить систему методом прогонки
     * @param matrix матрица коэффициентов
     * @param bias свободные коэффициенты
     * @return вектор решений
     */
    public static double[] solve(double[][] matrix, double[] bias) {
        int iters = 0;
        double[] result = new double[matrix.length];
        double[] P = new double[matrix.length];
        double[] Q = new double[matrix.length];

        P[0] = - matrix[0][2] / matrix[0][1];
        Q[0] = bias[0] / matrix[0][1];
        for (int i = 1; i < matrix.length; ++i) {
            P[i] = - matrix[i][2] / (matrix[i][1] + matrix[i][0] * P[i - 1]);
            Q[i] = (bias[i] - matrix[i][0] * Q[i - 1]) / (matrix[i][1] + matrix[i][0] * P[i - 1]);
            ++iters;
        }

        result[matrix.length - 1] = Q[matrix.length - 1];
        for (int i = matrix.length - 2; i >= 0; --i) {
            result[i] = P[i] * result[i + 1] + Q[i];
            ++iters;
        }

        System.out.println("Количество итераций: " + iters);

        return result;
    }

    /**
     * Протестировать решение
     * @param matrix матрица коэффициентов
     * @param bias свободные коэффициенты
     * @param result вектор решений
     * @return корректное/некорректное решение
     */
    public static boolean test(double[][] matrix, double[] bias, double[] result) {
        double lhs = matrix[0][1] * result[0] + matrix[0][2] * result[1];
        if (abs(lhs - bias[0]) >= EPS) {
            return false;
        }

        for (int i = 1; i < matrix.length - 1; ++i) {
            lhs = matrix[i][0] * result[i - 1] + matrix[i][1] * result[i] + matrix[i][2] * result[i + 1];
            if (abs(lhs - bias[i]) >= EPS) {
                return false;
            }
        }

        lhs = matrix[matrix.length - 1][0] * result[matrix.length - 2] + matrix[matrix.length - 1][1] * result[matrix.length - 1];

        if (abs(lhs - bias[matrix.length - 1]) >= EPS) {
            return false;
        }

        return true;
    }

    public static void main(String[] args) {
        double[][] matrix = new double[][] {
                {0, 10, -1},
                {-8, 16, 1},
                {6, -16, 6},
                {-8, 16, -5},
                {5, -13, 0}
        };

        double[] bias = new double[] {16, -110, 24, -3, 87};

        double[] result = solve(matrix, bias);

        System.out.println("Результат:");
        printVector(result);

        System.out.println("\nПроверка: " + test(matrix, bias, result));
    }
}
\end{minted}

\section*{Результат}

\begin{minted}{bash}
Количество итераций: 8
Результат:
1.0000 -6.0000 -6.0000 -6.0000 -9.0000 
Проверка: true
\end{minted}

\section*{Вывод}

В ходе лабораторной работы 1.2 мной был реализован алгоритм метода прогонки для решения СЛАУ 
с трехдиагональной матрицей. Программная реализация успешно решила поставленную задачу, 
продемонстрировав следующие результаты:

\begin{itemize}
\item Разработанная программа корректно обрабатывает входные данные в виде ненулевых элементов трехдиагональной матрицы и вектора правых частей
\item Реализованный алгоритм показал высокую эффективность при решении систем большой размерности
\item Вычисления выполняются за $O(n)$ операций, что подтверждает теоретические оценки вычислительной сложности метода
\end{itemize}

Для решения задачи были применены знания из курса вычислительной математики:
\begin{itemize}
\item Использован метод прогонки
\item Учтены условия диагонального преобладания для обеспечения устойчивости решения
\item Реализованы оптимальные схемы хранения данных (только ненулевые элементы)
\end{itemize}

Практическая реализация позволила глубже понять:
\begin{itemize}
\item Преимущества специализированных методов для разреженных матриц
\item Важность анализа устойчивости алгоритмов
\item Особенности работы с трехдиагональными матрицами в вычислительных задачах
\end{itemize}

Экспериментальные результаты подтвердили теоретические положения о том, 
что метод прогонки является оптимальным выбором для решения СЛАУ с трехдиагональными матрицами, 
сочетая вычислительную эффективность и надежность.

\pagebreak
