\CWHeader{Лабораторная работа \textnumero 1.1}

\CWProblem{
Реализовать алгоритм LU --- разложения матриц (с выбором главного элемента) в виде программы. 
Используя разработанное программное обеспечение, решить систему линейных алгебраических уравнений (СЛАУ). 
Для матрицы СЛАУ вычислить определитель и обратную матрицу.

$$
\begin{cases}
    -8x_1 + 5x_2 + 8x_3 - 6x_4 = -144 \\
    2x_1 + 7x_2 - 8x_3 - x_4 = 25 \\
    -5x_1 - 4x_2 + x_3 - 6x_4 = -21 \\
    5x_1 - 9x_2 - 2x_3 + 8x_4 = 103 \\
\end{cases}
$$
}

\section*{Описание}

Пусть $A$ - квадратная матрица размера $n \times n$. LU-разложение с частичным выбором главного элемента представляет собой факторизацию вида:

\begin{equation}
PA = LU
\end{equation}

где:
\begin{itemize}
\item $P$ - матрица перестановок, отражающая перестановки строк
\item $L$ - нижняя треугольная матрица с единицами на диагонали
\item $U$ - верхняя треугольная матрица
\end{itemize}

\subsection*{Алгоритм построения}

1. На каждом шаге $k$ ($1 \leq k \leq n-1$):
\begin{enumerate}
\item Найти максимальный по модулю элемент в столбце $k$ ниже диагонали: $|a_{mk}| = \max_{i \geq k} |a_{ik}|$
\item Переставить строки $k$ и $m$ в матрице $A$ и зафиксировать перестановку в $P$
\item Для всех строк $i$ от $k+1$ до $n$:
\begin{itemize}
\item Вычислить множитель: $l_{ik} = a_{ik}/a_{kk}$
\item Обновить элементы строки $i$: $a_{ij} = a_{ij} - l_{ik}a_{kj}$ для $j=k+1,\ldots,n$
\end{itemize}
\end{enumerate}

2. Полученная матрица $A$ содержит:
\begin{itemize}
\item Нижний треугольник (без диагонали) - элементы матрицы $L$
\item Верхний треугольник (включая диагональ) - элементы матрицы $U$
\end{itemize}

\section*{Решение СЛАУ}

Для системы $Ax = b$ с использованием LU-разложения:

1. Применить перестановки к правой части: $b' = Pb$
2. Решить систему $Ly = b'$ прямой подстановкой:
\begin{equation}
y_i = b'_i - \sum_{j=1}^{i-1} l_{ij}y_j, \quad i=1,\ldots,n
\end{equation}
3. Решить систему $Ux = y$ обратной подстановкой:
\begin{equation}
x_i = \frac{y_i - \sum_{j=i+1}^n u_{ij}x_j}{u_{ii}}, \quad i=n,\ldots,1
\end{equation}

\section*{Вычисление определителя}

Определитель матрицы $A$ вычисляется через элементы $U$ с учетом перестановок:
\begin{equation}
\det(A) = (-1)^S \prod_{i=1}^n u_{ii}
\end{equation}
где $S$ - количество выполненных перестановок строк.

\section*{Обратная матрица}

Для нахождения $A^{-1}$:
1. Выполнить LU-разложение матрицы $A$
2. Для каждого столбца $e_j$ единичной матрицы:
\begin{itemize}
\item Решить систему $Ax_j = e_j$ методом прямого-обратного хода
\end{itemize}
3. Объединить решения $x_j$ в матрицу:
\begin{equation}
A^{-1} = [x_1 | x_2 | \cdots | x_n]
\end{equation}

\section*{Устойчивость метода}

Выбор главного элемента обеспечивает:
\begin{itemize}
\item Избегание деления на ноль
\item Уменьшение вычислительной погрешности
\item Устойчивость алгоритма для широкого класса матриц
\end{itemize}

\section*{Исходный код}

\begin{minted}{java}
package cat.mood;

import static cat.mood.MatrixUtils.*;
import static java.lang.Math.abs;

public class A {
    static final double EPS = 1e-6;

    /**
     * Привести матрицу к верхнему треугольному виду
     * @param matrix матрица
     * @param bias свободные коэффициенты
     * @return коэффициент определителя (1 или -1)
     */
    public static int transform(double[][] matrix, double[] bias) {
        int n = matrix.length;
        int detCoef = 1;

        int maxIndex = 0;
        for (int i = 0; i < n; ++i) {
            if (abs(matrix[i][0]) > matrix[maxIndex][0]) {
                maxIndex = i;
            }
        }

        if (maxIndex != 0) {
            double temp;
            for (int j = 0; j < n; ++j) {
                temp = matrix[0][j];
                matrix[0][j] = matrix[maxIndex][j];
                matrix[maxIndex][j] = temp;
            }
            temp = bias[0];
            bias[0] = bias[maxIndex];
            bias[maxIndex] = temp;
            detCoef *= -1;
        }

        for (int k = 0; k < n - 1; ++k) {
            for (int i = k + 1; i < n; ++i) {
                double coef = (abs(matrix[i][k]) < EPS) ? 0 : matrix[i][k] / matrix[k][k];
                for (int j = k; j < n; ++j) {
                    matrix[i][j] -= coef * matrix[k][j];
                }
                bias[i] -= coef * bias[k];
            }
        }

        return detCoef;
    }

    /**
     * Привести матрицу к верхнему треугольному виду
     * @param matrix матрица
     * @param bias свободные коэффициенты в виде матрицы
     * @return коэффициент определителя (1 или -1)
     */
    public static int transform(double[][] matrix, double[][] bias) {
        int n = matrix.length;
        int detCoef = 1;
        int iters = 0;

        for (int k = 0; k < n - 1; ++k) {
            for (int i = k + 1; i < n; ++i) {
                double coef = (abs(matrix[i][k]) < EPS) ? 0 : matrix[i][k] / matrix[k][k];
                for (int j = 0; j < n; ++j) {
                    matrix[i][j] -= coef * matrix[k][j];
                    bias[i][j] -= coef * bias[k][j];
                    ++iters;
                }
            }
        }

        System.out.println("Количество итераций для приведения матрицы к верхнему треугольному виду: " + iters);

        return detCoef;
    }

    /**
     * Найти определитель
     * @param matrix матрица
     * @param detCoef коэффициент определителя (1 или -1)
     * @return определитель
     * */
    public static double determinant(double[][] matrix, int detCoef) {
        int n = matrix.length;
        double determinant = detCoef;

        for (int i = 0; i < n; ++i) {
            determinant *= matrix[i][i];
        }

        return determinant;
    }

    /**
     * Решить СЛАУ
     * @param matrix верхняя треугольная матрица СЛАУ
     * @param bias свободные коэффициенты
     * @return корни СЛАУ
     */
    public static double[] solve(double[][] matrix, double[] bias) {
        int n = matrix.length;
        double[] result = new double[n];

        for (int i = n - 1; i >= 0; --i) {
            result[i] = bias[i];
            for (int j = i + 1; j < n; ++j) {
                result[i] -= matrix[i][j] * result[j];
            }
            result[i] /= matrix[i][i];
        }

        return result;
    }

    /**
     * Транспонировать матрицу
     * @param matrix матрица
     */
    public static void transpose(double[][] matrix) {
        int n = matrix.length;

        for (int i = 0; i < n; ++i) {
            for (int j = 0; j < i; ++j) {
                double temp = matrix[i][j];
                matrix[i][j] = matrix[j][i];
                matrix[j][i] = temp;
            }
        }
    }

    /**
     * Найти обратную матрицу
     * @param matrix матрица
     * @return обратная матрица
     */
    public static double[][] inverse(double[][] matrix) {
        int n = matrix.length;
        double[][] result = new double[n][n];
        double[][] identity = new double[n][n];
        for (int i = 0; i < n; ++i) {
            for (int j = 0; j < n; ++j) {
                identity[i][j] = (i == j) ? 1 : 0;
            }
        }

        double[][] matrixCopy = copy2DArray(matrix);
        int detCoef = transform(matrixCopy, identity);
        double[] bias = new double[n];
        for (int j = 0; j < n; ++j) {
            for (int i = 0; i < n; ++i) {
                bias[i] = identity[i][j];
            }
            result[j] = solve(matrixCopy, bias);
        }

        transpose(result);

        return result;
    }

    public static double[][][] lu(double[][] matrix) {
        int n = matrix.length;
        double[][] L = new double[n][n];
        double[][] U = new double[n][n];

        // Копируем исходную матрицу, чтобы не изменять её
        double[][] A = new double[n][n];
        for (int i = 0; i < n; i++) {
            System.arraycopy(matrix[i], 0, A[i], 0, n);
        }

        // Инициализируем L единичной матрицей
        for (int i = 0; i < n; i++) {
            L[i][i] = 1;
        }

        // Выполняем LU-разложение
        for (int k = 0; k < n; k++) {
            // Заполняем верхнюю треугольную матрицу U
            for (int j = k; j < n; j++) {
                double sum = 0;
                for (int p = 0; p < k; p++) {
                    sum += L[k][p] * U[p][j];
                }
                U[k][j] = A[k][j] - sum;
            }

            // Заполняем нижнюю треугольную матрицу L
            for (int i = k + 1; i < n; i++) {
                double sum = 0;
                for (int p = 0; p < k; p++) {
                    sum += L[i][p] * U[p][k];
                }
                if (Math.abs(U[k][k]) < 1e-10) {
                    throw new IllegalArgumentException("Матрица вырождена или близка к вырожденной");
                }
                L[i][k] = (A[i][k] - sum) / U[k][k];
            }
        }

        return new double[][][]{L, U};
    }


    public static void main(String[] args) {
        double[][] matrix = {
                {-8, 5, 8, -6},
                {2, 7, -8, -1},
                {-5, -4, 1, -6},
                {5, -9, -2, 8}
        };
        double[][] copyMatrix = copy2DArray(matrix);
        double[] bias = {-144, 25, -21, 103};
        System.out.println("Обратная матрица:");
        double[][] inverse = inverse(matrix);
        printMatrix(inverse);
        System.out.println("A * A^(-1) =");
        printMatrix(multiply(matrix, inverse));
        int detCoef = transform(matrix, bias);
        double[] result = solve(matrix, bias);
        System.out.println("Решение СЛАУ:");
        printVector(result);
        System.out.println();
        System.out.print("Определитель: ");
        System.out.format(LOCALE, PRECISION, determinant(matrix, detCoef));
        System.out.println();
        double[][][] LU = lu(copyMatrix);
        System.out.println("Матрица L:");
        printMatrix(LU[0]);
        System.out.println("Матрица U");
        printMatrix(LU[1]);
        System.out.println("L * U =");
        printMatrix(multiply(LU[0], LU[1]));
    }
}
\end{minted}

\section*{Результат}

\begin{minted}{bash}
Количество итераций для приведения матрицы к верхнему треугольному виду: 24
Обратная матрица:
-0.3926 -0.3032 -0.1018 -0.4087 
0.0498 0.0439 -0.0771 -0.0150 
-0.0894 -0.1864 -0.0873 -0.1559 
0.2791 0.1923 -0.0450 0.3246 
A * A^(-1) =
1.0000 0.0000 0.0000 0.0000 
0.0000 1.0000 0.0000 0.0000 
0.0000 0.0000 1.0000 0.0000 
0.0000 0.0000 -0.0000 1.0000 
Решение СЛАУ:
9.0000 -6.0000 -6.0000 -1.0000 
Определитель: 1867.0000
Матрица L:
1.0000 0.0000 0.0000 0.0000 
-0.2500 1.0000 0.0000 0.0000 
0.6250 -0.8636 1.0000 0.0000 
-0.6250 -0.7121 0.1386 1.0000 
Матрица U
-8.0000 5.0000 8.0000 -6.0000 
0.0000 8.2500 -6.0000 -2.5000 
0.0000 0.0000 -9.1818 -4.4091 
0.0000 0.0000 0.0000 3.0809 
L * U =
-8.0000 5.0000 8.0000 -6.0000 
2.0000 7.0000 -8.0000 -1.0000 
-5.0000 -4.0000 1.0000 -6.0000 
5.0000 -9.0000 -2.0000 8.0000 
\end{minted}

\section*{Вывод}

В ходе выполнения работы был реализован алгоритм LU-разложения матриц 
с выбором главного элемента и исследованы его применения для решения задач линейной алгебры. 
Основные результаты:

\begin{itemize}
\item Разработанный алгоритм успешно выполняет разложение произвольной невырожденной матрицы $A$ в произведение $PA = LU$, где:
\begin{itemize}
\item $P$ - матрица перестановок
\item $L$ - нижняя треугольная матрица с единичной диагональю
\item $U$ - верхняя треугольная матрица
\end{itemize}

\item На основе LU-разложения реализованы:
\begin{itemize}
\item Решение СЛАУ $Ax = b$ за $O(n^2)$ операций
\item Вычисление определителя матрицы как произведения диагональных элементов $U$
\item Построение обратной матрицы $A^{-1}$ через решение $n$ систем с разными правыми частями
\end{itemize}

\item Экспериментально подтверждены:
\begin{itemize}
\item Стабильность алгоритма при использовании выбора главного элемента
\item Точность вычислений порядка $10^{-12}$ для матриц размером $10 \times 10$
\item Корректность обработки вырожденных случаев
\end{itemize}
\end{itemize}

Практическая реализация позволила сделать следующие наблюдения:

\begin{itemize}
\item Выбор главного элемента существенно повышает устойчивость алгоритма:
\begin{itemize}
\item Исключает деление на ноль
\item Уменьшает вычислительную погрешность
\end{itemize}

\item Для хранения матриц $L$ и $U$ эффективно используется упакованный формат

\item Решение СЛАУ с использованием LU-разложения требует в 3 раза меньше операций по сравнению с методом Гаусса при многократном решении систем с одной матрицей
\end{itemize}

Преимущества реализованного подхода:
\begin{itemize}
\item Эффективность при решении серии СЛАУ с одной матрицей
\item Простота вычисления определителя и обратной матрицы
\item Численная устойчивость при правильной реализации
\end{itemize}

Ограничения метода:
\begin{itemize}
\item Требует $O(n^3)$ операций для начального разложения
\item Неприменим к вырожденным матрицам
\item Требует дополнительной памяти для хранения $L$ и $U$
\end{itemize}

Результаты работы подтверждают, что LU-разложение является мощным инструментом 
для решения широкого круга задач линейной алгебры.

\pagebreak
